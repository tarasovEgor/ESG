% https://www.swrit.ru/doc/gost34/34.602-2020.pdf
% Сделано по гост 34.602-2020
{ % чтобы в главном тексте не сбрасывались заголовки
%%%%%%%%%%%%%%%%%%%%% переопределение секций чтобы было 1.1. вместо А.1.1.
\renewcommand*{\thesection}{\arabic{section}}
\titleformat{\section}{\large\bfseries}{\thesection.}{4pt}{}

\renewcommand*{\thesubsection}{\arabic{section}.\arabic{subsection}}
\titleformat{\subsection}{\large\bfseries}{\thesubsection.}{4pt}{}

\renewcommand*{\thesubsubsection}{\arabic{section}.\arabic{subsection}.\arabic{subsubsection}}
\titleformat{\subsubsection}{\normalsize\bfseries}{\thesubsubsection.}{4pt}{}
\setcounter{secnumdepth}{4}

%%%%%%%%%%%%%%%%%%%%%%%%%% 5
{ % для титульного листа
  \chapter*{ПРИЛОЖЕНИЕ А Техническое задание на разрабатываемую систему}\label{tz_chap}

\stepcounter{chapter}
\thispagestyle{empty}
\centering

\begin{flushright}
  \MakeUppercase{Утверждено}

  A.B.00001-01 ТЗ 01
\end{flushright}

\vfill

\textbf{\MakeUppercase{Система для автоматического сбора, анализа и визуализации информации по этичности компаний}}

\textbf{Техническое задание}

\textbf{\textit{Лист утверждения}}

\vbox{
  \parbox{6cm}{
    \begin{sideways}
      \setlength\arrayrulewidth{2pt}
      \begin{tabular}{|c|c|c|c|c|}
        \hline
        Инв. № подл. & Подпись и дата & Взам. инв. № & Инв. № дубл. & Подпись и дата \\
        \hline
                     &&&&\\
        \hline
      \end{tabular}
    \end{sideways}
  }
  \hfill
  \parbox{9.4cm}{
    \begin{flushright}
      \begin{minipage}[t]{0.4\textwidth}
        Руководитель разработки

        \vspace{3mm}
        \makebox[6cm][r]{\hrulefill~Бузмаков~А.В.}
        \makebox[6cm][r]{<<\rule{7mm}{0.4pt}>>\hrulefill~\the\year}
      \end{minipage}
    \end{flushright}
    \vspace{10mm}
    \begin{flushright}
      \begin{minipage}[t]{0.4\textwidth}
        Исполнитель

        \vspace{3mm}
        \makebox[6cm][r]{\hrulefill~Соломатин~Р.И.}
        \makebox[6cm][r]{<<\rule{7mm}{0.4pt}>>\hrulefill~\the\year}
      \end{minipage}
    \end{flushright}
  }
}
\newpage
}
\section{Общие сведения}
Наименование программы – <<Система для автоматического сбора, анализа и визуализации информации по этичности компаний>> (далее – <<Система>>). Основная функция системы - сбор и анализ данных из различных источников, включая новостные сайты, социальные сети, отзывы о компаниях и другие открытые источники данных. Система использует алгоритмы машинного обучения и обработки естественного языка для автоматической обработки данных и определения этичности компаний.

Система также предоставляет визуализацию данных в виде графиков и диаграмм, позволяя пользователям легко понять и сравнивать данные по разным компаниям. Кроме того, система может предоставлять аналитические отчеты и рекомендации по улучшению этичности компаний на основе собранных данных.

Система разрабатывается в рамках выполнения выпускной квалификационной работы. Основанием для разработки являются:
\begin{itemize}
  \item Положение о курсовой и выпускной квалификационной работе студентов, обучающихся по программам бакалавриата, специалитета и магистратуры в Национальном исследовательском университете <<Высшая школа экономики>>, утвержденным ученым советом НИУ ВШЭ (протокол от 28.11.2014 № 08), с изменениями от 29.03.2016;
  \item Правила подготовки выпускной квалификационной работы студентов основной образовательной программы бакалавриата <<Программная инженерия>> по направлению подготовки 09.03.04. Программная инженерия, утвержденные протоколом ученого совета НИУ ВШЭ – Пермь от 19.11.2020 № 8.2.1.7-10/10.
\end{itemize}

\section{Цели и назначение создания автоматизированной системы}

\subsection{Цели создания АС}
Целью создания системы является получение инструмента который позволит анализировать компании на основании их этичности, соответствующего следующим требованиям:
\begin{itemize}
  \item Показывать историю изменений индекса с возможностью фильтровать по
        \begin{itemize}
          \item годам;
          \item отраслям компаний, с возможностью множественного выбора;
          \item компаниям, с возможностью множественного выбора;
          \item моделям, с возможностью множественного выбора;
          \item источникам, с возможностью множественного выбора.
        \end{itemize}
  \item Агрегировать значения индекса по годам и кварталам;
  \item Анализировать тексты для построения индекса этичности;
  \item Иметь возможность добавления анализа текста несколькими вариантами;
  \item Сохранять тексты для последующего анализа другими методами;
  \item Система должна собирать данные с сайтов banki.ru, sravni.ru и комментарии из групп {}<<вконтаке>>{};
  \item На сайте должен быть график, который показывать изменение индекса этичности компаний и количества собранных отзывов по разным источникам.
  \item Для расчета индекса этичности компаний на основании рецензий должна использоваться формула\ref{eq:ethics_tz}:
        \begin{equation}
          \label{eq:ethics_tz}
          \begin{aligned}
            \text{Base index} &= \frac{\text{positive} - \text{negative}}{\text{positive} + \text{negative}} \\
            \text{Std index} &= \sqrt{\frac{\text{positive}}{\text{negative} \cdot (\text{positive} + \text{negative})^{3}} + \frac{\text{negative}}{\text{positive} \cdot (\text{positive} + \text{negative})^{3}}} \\
            \text{Index} &= ({2\cdot({\text{Base index}}-{\text{Mean index}} > 0) - 1})\cdot\\
                              &{max\left(\left|{\text{Base index}}-{\text{Mean index}}\right|-{\text{Std index}}, 0\right)}
          \end{aligned}
        \end{equation}

        \(positive\) -- количество позитивных предложений,

        \(negative\) -- количество негативных предложений,

        \(Mean\ index\) -- среднее значения для пар источник сбора данных и модели, которая обрабатывала предложения.
\end{itemize}


\subsection{Назначение АС}
Система предназначена для собора и анализа отзывы потребителей с различных веб-сайтов, с помощью алгоритмов обработки естественного языка.
\section{Характеристика объекта автоматизации}
Система автоматизирует процесс анализа этичности компаний.
\section{Требования к автоматизированной системе}
Требования к АС:
\begin{enumerate}
  \item построение графика не должно занимать больше секунды;
  \item данные должны собираться автоматически;
  \item данные должны обрабатываться автоматически;
  \item система должны способна работать с большим объемом информации;
  \item система должна быть стабильна.
\end{enumerate}
\subsection{Требования к структуре АС в целом}
Должно быть несколько модулей, которые общаются между собой с помощью HTTP API:
\begin{itemize}
  \item сборщики данных;
  \item взаимодействие с базой данных(API);
  \item сайт;
  \item модели для обработки данных.
\end{itemize}
Все подсистемы должны быть в Docker контейнерах.
\subsection{Требования к функциям (задачам), выполняемым АС}
\subsubsection{Требования к API}
Система взаимодействия с базой данных должна выполнять функции:
\begin{itemize}
  \item хранить информацию о компаниях;
  \item модель должна отдавать информацию о компаниях из разных сфер;
  \item хранить информацию о разных источниках;
  \item добавлять различные источники;
  \item получать отзывы и разбивать их на предложения;
  \item иметь возможность отдавать необработанные предложения в зависимости от модели;
  \item сохранять информацию о моделях;
  \item сохранять результат обработки предложений;
  \item агрегировать результат обработки моделей для каждого квартала;
  \item хранить информацию о состоянии сборщиков данных, если у них возникнут проблемы;
  \item отдавать информацию об индексе менее, чем за минуту;
  \item рассчитывать индекс на основе полученных результатов обработки предложений.
\end{itemize}
\subsubsection{Требования к сборщикам данных}
Сборщики данных должны выполнять функции:
\begin{itemize}
  \item собирать отзывы пользователей ежедневно;
  \item собранные отзывы отправлять на API.
\end{itemize}
\subsubsection{Требования к моделям}
Модели должны выполнять функции:
\begin{itemize}
  \item обрабатывать отзывы пользователей ежедневно;
  \item обработанные отзывы отправлять на API.
\end{itemize}
\subsection{Требования к видам обеспечения АС}
\subsubsection{Требования к лингвистическому обеспечению}
Система должна соответствовать следующим требованиям:
\begin{itemize}
  \item Программный код должен быть реализован на языке Python;
  \item Документация к программе должна быть на русском языке. Других языков не планируется.
\end{itemize}
\subsubsection{Требования к программного обеспечению}
Система должна использовать:
\begin{itemize}
  \item Для разработки API следует использовать библиотеку FastAPI;
  \item Для взаимодействия с базой данных должна использоваться библиотека SQLAlchemy, а для миграций Alembic;
  \item Сборщики данных должны собирать информацию с помощью библиотеки запросов
        requests и для работы с HTML Beautiful Soup;
  \item Для нейросетевых моделей должен использоваться Pytorch;
  \item Для клиентской части будет использоваться библиотека React.
\end{itemize}
\subsubsection{Требования к техническому обеспечению}
Для работы приложения необходим сервер, который обладает следующими параметрами:
\begin{itemize}
  \item Процессор с тактовой частотой не ниже 3 ГГц и 8 ядрами;
  \item Операционная система Ubuntu 22.04 и выше;
  \item Docker 20.10.23 и выше.
  \item Графическая карта NVIDIA GTX 3080;
  \item Оперативная память не менее 16 Гб;
  \item Доступное место на жестком диске не менее 100 Гб;
\end{itemize}
\subsubsection{Требования к информационному обеспечению}
Система должна соответствовать следующим требованиям:
\begin{itemize}
  \item Система должна использовать PostgreSQL;
  \item Сервисы между собой должны взаимодействовать при помощи HTTP;
  \item Для базы данных должен всегда быть резервная копия данных.
\end{itemize}

\subsection{Общие технические требования к АС}
\subsubsection{Требования к численности и квалификации персонала}
Для разработки системы требуется программист со средней квалификацией. Для
работы с конечной системой (сайтом), не требуется высокой квалификации, поэтому
пользователь с ней справится пользователь, который пользуется сайтами.
\subsubsection{Требования к надежности}
Надежность системы зависит от надежности функционирования сервера. Устойчивое
функционирование программы будет обеспечено с помощью:
\begin{itemize}
  \item Бесперебойное питание сервера;
  \item Использованием лицензионного программного обеспечения, необходимого для
        запуска приложения, включая лицензионную операционную систему;
  \item Регулярным выполнением рекомендаций Министерства труда и социального развития РФ, изложенных в Постановлении от 23 июля 1998 г. <<Об утверждении межотраслевых типовых норм времени на работы по сервисному обслуживанию ПЭВМ и оргтехники и сопровождению программных средств>>;
  \item Регулярным выполнением требований ГОСТ 51188-98 <<Защита информации. Испытания программных средств на наличие компьютерных вирусов>>;
\end{itemize}
Время восстановление системы будет до 10 минут с момента сбоя.
\subsubsection{Требования по сохранности информации при авариях}
Отказы как самой системы, так и ее отдельных функций, могут привести к
аварийному завершению работы программы, однако при перезапуске программы ее
функциональность не должна пострадать. При таких сбоях программы база данных не
должна пострадать. Дополнительно все данные будут резервно копироваться на дополнительную базу данных.
\section{Состав и содержание работ по созданию автоматизированной системы}
\begin{longtblr}
  [caption={Этапы реализации, контрольные точки проекта},]
  {
    colspec = {|X|X|X|},
    rowhead = 1,
    hlines,
  }
  % BEGIN RECEIVE ORGTBL stages
\textbf{Основной этап} & Подэтап & \textbf{Крайний срок}\\[0pt]
Анализ & Литературный обзор & 26.12.2022\\[0pt]
Анализ & Сравнительный анализ существующих решений & 08.01.2023\\[0pt]
Анализ & Анализ сценариев использования & 14.01.2023\\[0pt]
Анализ & Написание тех задания & 21.01.2023\\[0pt]
Проектирование & Проектирование архитектуры приложения & 14.02.2023\\[0pt]
Проектирование & Проектирование базы данных & 20.03.2023\\[0pt]
Проектирование & Проектирование графического интерфейса & 20.03.2023\\[0pt]
Проектирование & Проектирование алгоритмов машинного обучения & 20.03.2023\\[0pt]
Разработка & Разработка алгоритмов для анализа текста & 01.05.2023\\[0pt]
Разработка & Реализация серверной части & 01.05.2023\\[0pt]
Разработка & Реализация клиентской части & 01.05.2023\\[0pt]
Тестирование & Подготовка тестовых сценариев & 15.05.2023\\[0pt]
Тестирование & Функциональное тестирование & 15.05.2023\\[0pt]
Тестирование & Системное тестирование & 15.05.2023\\[0pt]
Завершение & Сдача проекта & 22.05.2023\\[0pt]
% END RECEIVE ORGTBL stages
\end{longtblr}
\begin{comment}
  #+ORGTBL: SEND stages orgtbl-to-latex
  | *Основной этап* | Подэтап                                      | *Крайний срок* |
  | Анализ          | Литературный обзор                           |     26.12.2022 |
  | Анализ          | Сравнительный анализ существующих решений    |     08.01.2023 |
  | Анализ          | Анализ сценариев использования               |     14.01.2023 |
  | Анализ          | Написание тех задания                        |     21.01.2023 |
  | Проектирование  | Проектирование архитектуры приложения        |     14.02.2023 |
  | Проектирование  | Проектирование базы данных                   |     20.03.2023 |
  | Проектирование  | Проектирование графического интерфейса       |     20.03.2023 |
  | Проектирование  | Проектирование алгоритмов машинного обучения |     20.03.2023 |
  | Разработка      | Разработка алгоритмов для анализа текста     |     01.05.2023 |
  | Разработка      | Реализация серверной части                   |     01.05.2023 |
  | Разработка      | Реализация клиентской части                  |     01.05.2023 |
  | Тестирование    | Подготовка тестовых сценариев                |     15.05.2023 |
  | Тестирование    | Функциональное тестирование                  |     15.05.2023 |
  | Тестирование    | Системное тестирование                       |     15.05.2023 |
  | Завершение      | Сдача проекта                                |     22.05.2023 |
\end{comment}

\section{Порядок разработки автоматизированной системы}
В разделе «Порядок разработки автоматизированной системы» приводят следующее:
\begin{itemize}
  \item порядок организации разработки АС;
  \item перечень документов и исходных данных для разработки АС;
  \item перечень документов, предъявляемых по окончании соответствующих этапов работ;
  \item порядок проведения экспертизы технической документации;
  \item перечень макетов (при необходимости), порядок их разработки, изготовления, испытаний, необходимость разработки на них документации, программы и методик испытаний;
  \item порядок разработки, согласования и утверждения плана совместных работ по разработке АС;
  \item порядок разработки, согласования и утверждения программы работ по стандартизации;
  \item требования к гарантийным обязательствам разработчика;
  \item порядок проведения технико-экономической оценки разработки АС;
  \item порядок разработки, согласования и утверждения программы метрологического обеспечения, программы обеспечения надежности, программы эргономического обеспечения.
\end{itemize}

\section{Порядок контроля и приемки автоматизированной системы}
Осуществление приемо-сдаточных испытаний для всей системы осуществляется на основе Программы и методики испытаний и включает:
\begin{itemize}
    \item Функциональное тестирование;
    \item Тестирование удобства эксплуатации;
    \item Оценка сгенерированных уровней.
\end{itemize}

\section{Требования к составу и содержанию работ по подготовке объекта автоматизации к вводу автоматизированной системы в действие}
\begin{longtblr}[caption={Требования к программной документации}]{colspec = {|X|X|},rowhead = 1,hlines,}
    \textbf{Название документа }          & \textbf{Краткое содержание} \\
    Текст программы \par(ГОСТ 19.401–78)  & Программный код всех модулей программы с необходимыми комментариями. \\
    Программа и методика испытаний \par(ГОСТ 19.301–79) & Требования, подлежащие проверке при испытании программы, а также порядок и методы их контроля. \\
    Техническое задание \par(34.602-2020) & Назначение и область применения программы, технические, технико-экономические и специальные требования, предъявляемые к программе, необходимые стадии и сроки разработки, виды испытаний. \\
\end{longtblr}
}
